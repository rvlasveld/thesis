% !TEX root = ../main.tex
% Appendix F

\chapter{Afronden} % Main appendix title

\label{AppendixF} % For referencing this appendix elsewhere, use~\ref{AppendixA}

\lhead{Appendix F. \emph{Afronden}}

\begin{itemize}
  \item Typo's checken en fixen
  \item Chapter 2: Literature review
  \begin{itemize}
    \item Prima hoeveelheid, niet meer veel bij doen
    \item TODO's: alleen aan elkaar lijmen, geen nieuwe info of details
    \item Onderwerpen schrappen: "statistical model/background": mag korter en kleiner
    \item Indeling finalizen. Is niet zo belangrijk.
  \end{itemize}
  \item Chapter 3: inhoud is prima
  \begin{itemize}
    \item fig 3.4: laten staan. Relateren aan andere figuur, zelfde symbolen gebruiken
    \item TODO: alleen aan elkaar lijmen, geen nieuwe inhoud. Eventueel kort refereren naar bron voor meer duiding (parameters C etc)
    \item §3.5 en §3.6: overslaan, komt in §5 en §6.
  \end{itemize}
  \item Chapter 4:
  \begin{itemize}
    \item In begin aangeven wat mijn contributie is: * continue data met changes erin (gelabeled), en * toepassen van oc-svm op HAR data.
    \item Aangeven wat contributie in "proposed method" is: aangeven wat het zelfde en verschilt van Camci/Takeuchi. O.a.: toepassing op real-world data.
    \item Aangeven geen aparte "membership" check zoals Camci doet --> argumenteer robustheid (outliers e.d.).
    \item Naam van hoofdstuk: "Application to real-world HAR data" o.i.d. (past beter met pre-post processing sections). Meer focus op hele systeem, niet de algoritmes.
    \item Naam bedenken voor toepassing, om makkeljk te kunnen refereren.
  \end{itemize}
  \item Chapter 6:
  \begin{itemize}
    \item Foto's van recordings (veel) kleiner en minder. Alleen om setting te duiden.
    \item Een enkele plot van hele run. Verder kleine stukjes uit plots om interessante stukken te duiden (overgang lopen/rennen).
    \item Stukjes van plots gebruiken om verschillen te duiden.
    \item Echte resultaten in (box)plots en tabellen zetten. "Benefit" en FAR.
    \item Performance: vooral aangeven dat het nieuwe (subjectieve) data is. Voor echte vergelijkingen wijzen naar §5.
    \item Bijzondere mis-detecties (of extra detecties) aan het einde duiden ("remarks").
    \item Overall: het moet wat "cleaner"
  \end{itemize}
  \item Chapter 7: prima, aanvullen op de open plekken
\end{itemize}

Tussendoor:
\begin{itemize}
  \item Verantwoording SVM: generieke methode die op veel types data is gebruikt. "Hypothese" van onderzoek is: OC-SVM werkt op HAR data. Terug laten komen in §1 introduction en §7 conclusie.
\end{itemize}

\section{Prioriteiten}
\begin{enumerate}
  \item Hoofstuk 1, introduction
  \item Abstract
  \begin{itemize}
    \item Vraagstelling
    \item Antwoord
    \item Aanleiding tot onderzoek
    \item Samenvatting van onderzoek
    \item Wat gedaan is: Praktische toepassen en uitwerking OC-SVM
    \item Wat bijzonder is: applicatie in domein
  \end{itemize}
  \item Lijstje hierboven afwerken
  \item TODO's eruit filteren
  \item Resultaten chapter 6
  \item Conclusie afmaken
\end{enumerate}

\section{Planning}
\begin{itemize}
  \item Deadline: 10 januari, geprint op bureau Anne. Final versie geen todo's meer.
\end{itemize}