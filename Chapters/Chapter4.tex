% !TEX root = ../main.tex
% Chapter 4

\chapter{Application to Human Activity Time Series Data}

\label{Chapter4} % For referencing the chapter elsewhere, use~\ref{Chapter4}

\lhead{Chapter 4. \emph{Application to Human Activity Time Series Data}} % This is for the header on each page - perhaps a shortened title

\TODO{Decide on name}
\begin{itemize}
  \item Human Activity Support Vector Change Detection
  \item Inertial Sensor Time Series Svm Temporal Segmentation
  \item Support Vector bases Human Activity Temporal Segmentation
  \item Human Activity Temporal Segmentation by Support Vector Machines: HATS-SVM
  \item HATS with One-Class SVM: HATS-OCS, HATS-OS, HATSOS
  \item OCS-HATS
  \item HATS-SVDD
\end{itemize}


\begin{figure}
  \centering
    \includegraphics[width=\textwidth]{./Figures/chapter4/method_setup.pdf}
  \caption[Method setup]{Schematic overview of the change detection method. The first step is the data gathering, described in Section \ref{sec:method_data_gathering}. Ater the pre-processing, the data is used to construct a \gls{svdd} model, as described in Section \ref{sec:method_model_construction}. Section \ref{sec:method_model_features} describes which features of the model are used for the final change indication algorithm, discussed in Section \ref{sec:method_change_detection}.}
  \label{fig:method_overview}
\end{figure}

This chapter will introduce our method and setup for the temporal segmentation of human activities, using an \gls{oc-svm} approach.
It shows our method to apply a \gls{oc-svm} based temporal segmentation method to inertial sensor time series data, often used in the context of \acrlong{har}.
As far as we know it is the first in its kind, especially on the application of \gls{svdd} and \gls{svcpd} to real-world \gls{har} data.
The setup follows the structure as illustrated in \Cref{fig:method_overview} and further discussed in \Cref{sec:method_overview}.
It starts with the data collection phase, in which we both use artificial and real-world data.
The data sets are discussed in \Cref{sec:data_gathering}.
After (optional) pre-processing, the \gls{svdd} algorithm is used to create a model representation, which is iteratively updated.
In \Cref{sec:method_model_construction} this construction phase is discussed and shows the windowed approach.
From this constructed model we can extract some features.
In our case, we use the approximated radius $R$ of the hypersphere to detect changes.
\Cref{sec:method_model_features} goes into detail about the rationale of using that feature.
Finally, in \Cref{sec:method_change_detection} we discuss two different, but simple, methods to detect changes in the time series data, based on the extracted radius $R$.

% !TEX root = ../../main.tex
\section{Algorithm overview}\label{sec:method_overview}
The proposed method of this thesis follows the unifying framework as introduced by Takeuchi and Yamanishi \cite{takeuchi2006unifying} and an similar implementation by Camci \cite{camci2010change} with \glspl{svm}.
The unifying framework relates the detection of outliers with change points and divides the whole process in two stages.
The first stage determines the outliers in a time series by giving a score based on the deviation from a learned model, and thereby creates a new time series.
The second stage runs on that new created time series and calculates a average over a window of the outlier scores.
The problem of change detection is then reduced to outlier detection over that average-scored time series.
This method is named \gls{changeFinder} by the authors.
The implementation by Camci, which uses \glspl{svm} to detect changes is named \acrlong{svcpd}.

Whereas \gls{changeFinder} uses double probability estimation algorithm, our approach follows \gls{svcpd} by constructing a \gls{svm} over a sliding window.
The \gls{svcpd} algorithm uses the location of new data points in the feature space $\mathcal{F}$ with respect to the hypersphere and the hypersphere's radius $R$ to determine whether the new data point represents a change point.
Our approach is a slight simplification of the algorithm: we only use the updated model properties (\ie the radius $R$) to detect a change.
This difference is further discussed and justified in \Cref{sec:method_change_detection}.

This section gives a description of the method used for the experiments and change detection mechanism.
First described is the method to process the gathered sensor data.
A schematic overview is given in \Cref{fig:method_overview} and shows the steps of the method.
A more detailed explanation of the ``Update model'' step follows.

As graphically represented in \Cref{fig:method_overview}, the change detection method starts by processing the data from sensor, such as the accelerometer, magnetic orientation and rotation metrics.

The first step is to process the raw streams of data originating from a multiple of sensors.
The two processes applied are alignment and normalization.
Due to noisy sampling, not all the timestamps in the data streams are sensed at the same timestamp.
Since the \gls{svdd} method requires all the data stream at every timestamp and can not handle missing data on one of the timestamps, all the unique timestamps are filtered out.
Whilst this results in an overall filtering effect, in practice between $1\%$ and $5\%$ of each data stream is disregarded.
The effect of this filtering is not significant and the data is not modified.

Due to the nature of the sensor signals, a normalization step is required in order to set the weight for all the data streams equal.
The range of the accelerometer signal typically spans $-20$ to $20$, the magnetic field from $-60$ to $60$ and the rotations range is from $-1$ to $1$.
This means that a relative small change in the accelerometer stream could have a much larger impact on the model than the same (absolute) change in the rotation stream, whilst the latter has a larger relative impact.
The normalization step ensures that all data is weighted equally and changes in the data are all proportional.

In step $3$ the \gls{svdd} model is initialized.
The first full window over the data stream is used to construct an initial model.
During the initialization the parameters for the \gls{svdd} are provided, begin the kernel type (radial), \gls{rbf} with $\sigma$ and the outlier-fraction $C$.

Step $4$ is executed for every step-size $s$ data points in the stream.
Every update the oldest $s$ data points are removed from and $s$ new data points are added to the \gls{svdd} model.
The \gls{isvdd} algorithm by Tax and Duin~\cite{tax2002uniform} is used for efficient incremental model updating.
The model is (partially) reconstructed and new model properties, such as the radius of the hypersphere, is the result of this step \footnote{Other measures are also possible, for instance the distance from all the outliers to the boundary of the hypersphere or the number of outliers}.

This final step of this method, step $5$, is the interpretation of the model properties.
Many algorithms can be used for this process, all which take a one-dimensional time series as input and determine where change has occured.
In our setup we used the \gls{rt} and \gls{cusum} methods, to show the modularity of this step.
This final step is further discussed in \Cref{sec:change_detection_methods}.
% !TEX root = ../../main.tex
\section{Data Gathering}\label{sec:method_data_gathering}
In this section we briefly discuss the different data gathering methods used for the change detection algorithms and experiments.
\Cref{subsec:data_gathering_artificial} discusses the artificial data sets we will use.
In \Cref{subsec:data_gathering_real_world} an overview of the real-world data sets used is provided.
Both sections refer to Chapters~\ref{Chapter5} and~\ref{Chapter6} for more details, respectively.

% ------
\subsection{Artificial data}\label{subsec:data_gathering_artificial}
In order to provide an objective comparison to other methods, we will use artificial data sets which are also used in the researches on which our method is based.
These are the data sets used by Takeuchi and Yamanishi \cite{takeuchi2006unifying} and Camci \cite{camci2010change}.
Both construct a collection of one-dimensional time series data according to a second order \gls{ar} model:
\begin{equation}
  x_t = a_1 x_{t-1} + a_2 x_{t-2} + \epsilon_t.
\end{equation}
Over the different data series the mean and variance of the Gaussian random variable $\epsilon_t$ differs and changes at pre-determined change points.
Using this data set an objective quality measure over the change detection methods can be obtained and compared.
All the used data sets are listed and analyzed in \Cref{Chapter5}.

% ------
\subsection{Real-world data}\label{subsec:data_gathering_real_world}
In the second type of data sets we apply our method of change detection and temporal segmentation to real-world data sets.
For our setup we record the activities of humans performed both in- and outdoor in an uncontrolled environment.
Activities performed include sitting, standing, walking, running in a straight and curved line, and walking up- and downstairs.
Our method uses the signals from the accelerometer, magnetic field, and rotation sensors.
These time series data are used to detect change points.
A video recording from the performed activity is used to annotate the time series with real change points.
The discovered change point are compared with these annotated change points to give a subjective quality measure.
In \Cref{Chapter6} we give a detailed analysis of the performed activities and the recorded data sets.

For the experiments we used a HTC Sensation XE smartphone as recording device.
The activities were recorded using a free Android application \cite{sensorlogger}.
This application was chosen for its convenient data format of the sensor recording and its regularity of the sampling interval.
Table~\ref{tab:recorded_metrics} lists all the recorded metrics.
For our expiments we used the data for the accelerometer, magnatief field and rotation.

\begin{center}\begin{table}
  \begin{tabulary}{\textwidth}{|l|L|c|c|}
    \hline
    Metric & Description & Units of measure & Typical range \\
    \hline \hline
    Accelerometer & Acceleration force along each axis (including gravity). & $m/s^2$ & $-20$ -- $20$ \\
    \hline
    Gravity & Force of gravity along each axis. & $m/s^2$ & $-10$ -- $10$\\
    \hline
    Gyroscope & Rate of rotationg around each axis. & $rad/s$ & $-15$ -- $15$\\
    \hline
    Light & Light sensitive sensor at the front of the phone. & & $0$ -- $10000$ \\
    \hline
    Linear acceleration & Acceleration force along each axis (excluding gravity). & $m/s^2$ & $-20$ -- $20$ \\
    \hline
    Magnetic field & Geomagnetic field strength along each axis. & $\mu T$ & $-60$ -- $60$ \\
    \hline
    Orientation & Degrees of rotation around the three physical axis. & Degrees & $-100$ -- $360$ \\
    \hline
    Rotation & Measre of rotation around the device's rotation axis. & Unitless & $-1$ -- $1$\\
    \hline
  \end{tabulary}
  \caption[Measured metrics]{Measured metrics. The set of axis is always the triple (x, y, z) direction.}
  \label{tab:recorded_metrics}
\end{table}\end{center}

% !TEX root = ../../main.tex
\section{Model Construction: Incremental SVDD}\label{sec:method_model_construction}
After the data is collected and pre-processed (or in the case of the artificial data sets: generated), we construct an online incremental sliding window model construction algorithm.
We follow the method and implementation introduced by Tax and Laskov \cite{tax2003online}, the \acrlong{isvdd} method.
This method combines the techniques of online, unsupervised and incremental learning methods with the earlier introduced \gls{oc-svm} algorithm \gls{svdd}.
The method is first initialized with a window length and then in every step a new data object is added to and the last data object is removed from the working set.

Using the following abstract form of the \gls{svm} optimization problem, the extension of the incremental \gls{svm} to the \gls{svdd} can be carried out:
\begin{equation}
  \operatorname*{max}_\mu \operatorname*{min}_{\substack{
    0 \le x \le C \\
    \vectorsym{a}^T \vectorsym{x} + b = 0}
  } : W = -\vectorsym{c}^T\vectorsym{x} + \frac{1}{2}\vectorsym{x}^T K\vectorsym{x} + \mu(\vectorsym{a}^T\vectorsym{x} + b),
\end{equation}
where $\vectorsym{c}$ and $\vectorsym{a}$ are $n \times 1$ vectors, $K$ is a $n \times n$ matrix and $b$ is a scalar.
The \gls{svdd} implementation of this abstract form is set by the parameters $\vectorsym{c}=\operatorname*{diag}(K)$, $\vectorsym{a} = \vectorsym{y}$ and $b=1$.
The procudure for the incremental version has two operations: adding and removing a data object $k$.
When a data object $k$ added, its weight $x_k$ is initially set to $0$.
In case of an object removal, the weight is forced to be $x_k=0$.
Both the operations conclude with the recalculation $\mu$ and the weights $\vectorsym{x}$ for all the objects, in order to obtain the optimal solution for the enlarged or reduced data set.
The incremental learning algorithm follows from these two operations: new data objects are added to and old data objects are removed from the working set.

The size of the initial window of data objects has a lower bound determined by the hyperparameter $C$ (Equation \ref{eq:svdd_objective}).
Because of the equality constraint $\sum_{i=1}^n a_i x_i = 1$ and the box constraint $0 \le x_i \le C$, the number of objects in the working set must be at least $\ceil{\frac{1}{C}}$.
\TODO{How to ensure large enough window/working set}
% !TEX root = ../../main.tex
\section{Model Features}\label{sec:method_model_features}
In the previous section we have discussed the \gls{isvdd} method, which creates a \gls{oc-svm} representation of a working set of data objects at every step of the algorithms loop.
This section shows how we interpret the constructed model and extract features to obtain a measure which can be used for a indication of change points.
The following section discusses how this obtained measure is used to indicate change.

The \gls{isvdd} algorithm creates a spherical \gls{oc-svm} representation of the working set at every step of the algorithm.
This model is obtained by the minimization of Equation~\ref{eq:svdd_objective}, which incorporates the radius $R$ of the sphere and the distances $\vectorsym{\xi}$ from the outliers to the boundary.
We will use the the radius $R$ of the hypersphere as an indication of change.

In \cite{tax2002uniform} Tax and Duin provide an analysis of the error of the \gls{svdd} algorithm.
This error is based on
\begin{inparaenum}[\itshape 1\upshape)]
\item the fraction $f_{T-}$ of target objects that is rejected, and
\item the fraction $f_{O+}$ of outliers that is accepted.
\end{inparaenum}
Since in \gls{occ} situation typically there are (almost) no examples of outlier objects, Tax and Duin construct a method to generate outliers based on the assumption that the outliers are uniformly distributed around the target set.
To minimize the error, calculated by the fractions $f_{T-}$ and $f_{O+}$, we should minimize the volume of the target data description (\ie the boundary of \gls{svdd}).
This is because the fraction of accepted outliers $f_{O+}$ is an estimate of the volume of the target data description, with respect to the volume of the outlier distribution.
Tax and Duin provide a method to optimize the parameters of the \gls{svdd} method, \ie the trade-off parameter $C$ and the \gls{rbf} kernel width $\sigma$.
This optimization will result in the modification of the radius $R$ of Equation~\ref{eq:svdd_objective} and affects the Lagrangian inequality~(\ref{eq:svdd_inequality}).

Since in our method the parameters $C$ and $\sigma$ are kept constant, and thereby also the fraction of target objects being rejected, the only free parameter is the radius $R$.
During the \gls{isvdd} algorithm the volume of the hypersphere is being minimized.
This means that only the radius of the sphere will change during the algorithm.
We will interpret changes in the radius $R$ with the same continuity assumptions as discussed in Section~\ref{subsec:kernel_function}.
This means that when two objects (in this case radius sizes) are close to each other, their original objects are also close.
In other words, when the underlying distribution of time series is changed, and thus the characteristics of the data objects changes, the corresponding radius $R$ will also change.
Since in that case the data objects in the working set become more heterogeneous, the radius of the hypersphere will increase.
When, instead, the data objects change from a heterogeneous set to a more homogeneous, we expect the radius to decrease in value, since the data objects are closer to each other in the feature space.

With the \gls{isvdd} algorithm we have effectively implemented a form of dimensionality reduction by feature extraction, using the radius $R$.
The following section discusses the algorithms which can be applied to the extracted radius $R$ as a volume estimate.
We thereby follow the setup of the unifying framework by Takeuchi and Yamanishi~\cite{takeuchi2006unifying}, of which this section described the first stage.

\TODO{Put the SVM method more in the light of dimensionality reduction. Applies to text in whole thesis...}

% !TEX root = ../../main.tex
\section{Change Detection}\label{sec:method_change_detection}
In the previous sections we have discussed what data is used for the model construction, and how that model is interpreted in the context of change detection.
We have shown how the multi-dimensional signal is reduced to a single dimension time series data, based on the extracted radius $R$ from the constructed hypersphere.
This metric is used to discover changes in the underlying data generation process.
We argued that in the case the data objects set becomes more heterogeneous, we expect an increase in the radius $R$.
In the same manner we expect that when the change is in the oldest data objects of the working set, the radius $R$ will decrease, since the data becomes more homogeneous.
An abstract representation of these expectations is illustrated in \Cref{fig:radius_expectation}.

In this section we discuss the methods applicable to the extracted radius $R$ of the hypersphere, in order to discover a change in the underlying distribution properties.
It is the second stage of our algorithm, analogue to the second stage of the unifying framework by Takeuchi and Yamanishi~\cite{takeuchi2006unifying}.
It is also based on the \gls{svcpd} method by Camci~\cite{camci2010change}, although we have simplified the algorithm.

In the \gls{svcpd} the first step for a new data point $\vectorsym{z}$ is to check whether the data point lies within the enclosing (hyperspherical) boundary of the constructed model.
This will label the new data point as an in- or outlier, relative to the current data set.
Depending on that outcome, the model is updated to include the new data point.
The second step is to inspect the (changed) model properties, \ie the radius $R$.
In \gls{ocs-hats} we only employ the second step: the inspection of the radius $R$.
We regard the first step (determine if $\vectorsym{z}$ is an outlier) to be superfluent: in case it is an outlier, the model update step will change the radius $R$ in order to let the boundary include the new data point $\vectorsym{z}$.
Would the new data object $\vectorsym{z}$ be already an inlier, then the updated model would not have a significant change in radius $R$.
Our algorithm is outlined in \Cref{alg:ocs-hats}.

\begin{algorithm}
\caption{\acrlong{ocs-hats} algorithm.}
\label{alg:ocs-hats}
\begin{algorithmic}[1]
\Require time series $X$, window size $n$, thresholds $th_{low}$ and $th_{high}$, proximity factor $\delta$
\Ensure collection $change\_points$
\item[]
\State $change\_points \leftarrow$ empty collection
\State $radii \leftarrow$ empty collection
\State Initialize $w \leftarrow incremental\_svdd$ with first $n$ samples from $\vectorsym{x}$
\item[]
\While{ new sample $x$ from $X$ at time $t$}
  \item[]
  \State Add $x$ to $w$
  \State $sphere \leftarrow update\_hypersphere(w)$
  \State $radius \leftarrow extract\_radius(sphere)$
  \State $radii[t] \leftarrow radius$
  \item[]
  \State $\hbar \leftarrow$ calculate ratio from previous average $radii$ and current $radius$
  \item[]
  \If{ $\hbar$ \textless \, $th_{low}$ \textbf{or} $\hbar$ \textgreater \, $th_{high}$} %\Comment{Compare $\hbar$ with thresholds}
    \State Add current time $t$ to $change\_points$
    \State Empty previous $radii$
  \EndIf
  \item[]
  \State Remove oldest $x$ from $w$
\EndWhile

\item[]

\State Merge $\vectorsym{cp}$ based on proximity $\delta$
\end{algorithmic}
\end{algorithm}

The problem of change detection in complex multi-dimensional time series is now reduced to finding change in a one-dimensional time series.
Besides the dimensionality reduction, the form of the signal is also simplified.
Whereas the original signal was represented as a sinusoidal or second order \gls{ar} model, the new time series is of much simpler form.
In our method we leave open the precise method which is used to find change points in the obtained change indication values.
Below, we will gives two examples of methods which can be applied.
The first is a family of methods based on the \gls{cusum} method.
The second example is a ratio-based thresholding mechanism used by Camci~\cite{camci2010change}.
We conclude this section by discussing some post-processing techniques we have used, to decrease the number of false positives.

\subsection{CUSUM based methods}
Many of the simple change detection methods are based on \gls{cusum}, originally introduced by Page~\cite{page1954continuous}.
Other variations and extensions of the method are proposed and used (\cite{inclan1994use,alippi2006adaptive,hsu2007mosum}).
Here we will discuss the simple form, which only detects changes as an increase in value of the examples, in this case the radius $R$.
The cumulative sum for all the values is calculated:
\begin{equation}
  S_n = \sum_{k=0}^n R_k
\end{equation}
and a change is detected when the value of the cumulative sums minus the minimum encountered exceeds a predetermined threshold $h$:
\begin{equation}
  \left\{ S_n - \operatorname*{min}_{0 \le i < n} S_i \right\} \ge h.
\end{equation}

Extensions of the \gls{cusum} method have been proposed.
Amongst others, extensions which extend its use to two-threshold methods, and methods appropriate for changes in mean or variance \cite{inclan1994use}.
Whilst adaptive methods have been proposed \cite{alippi2006adaptive}, most methods require manual determination of the threshold parameter $h$.

\subsection{Ratio-thresholding}\label{subsec:ratio_thresholding}
The second example of an algorithm that interprets the change indication time series and transforms it to change detection is the radius ratio thresholding method used by Camci~\cite{camci2010change}.
The method shares characteristics with \gls{cusum}, since it relies on summation of historic data and thresholds for change detection.
The method calculates the radius ratio $\hbar_t$ by taking the ratio of the current radius $r_t$ at time $t$ to the average radii since the last change point $y$:
\begin{equation}\label{eq:ratio_radius}
  \hbar_t = \frac{r_t}{\operatorname*{mean}(r_{y:t-1})},
\end{equation}
where $\operatorname*{mean}(r_{y:t-1})$ is the average of the previous approximated radii.
By using only historic data, this method can be incorporated in an online change detection method.
The ratio $\hbar_t$ is compared with two thresholds: the low threshold $th_\text{low}$ and the high threshold $th_\text{high}$.
When the data objects in the working set become more homogeneous at time $t$, the radius $R$ of the hypersphere and thereby the ratio $\hbar_t$ will decrease.
When the new value of $\hbar_r$ is lower than the low threshold $th_\text{low}$, a change detection is observed for time $t$.
The same holds for a more heterogeneous set of data and an increase of $\hbar_r$ higher than $th_\text{high}$.
This is illustrated in \Cref{fig:tresholding}.

The appropriate values for the thresholds strongly depend on the characteristics of the data and the parameter $C$ (as discussed in \Cref{subsec:oc-svm-svdd}).
This trade-off parameter regulates the fraction of data objects that will be rejected from the constructed model.
Since a high value of $C$ assigns a high penalty to outliers, data objects are more likely to be incorporated into the hypersphere by increasing the radius.
For low values of $C$ the cost of rejecting data objects is, compared to the benefits of a smaller hypersphere, relatively low.
This gives a relation between the parameters $C$, $th_\text{low}$, $th_\text{high}$, and the sensitivity of the radius ratio based thresholding method.
A high value of $C$, or values close to $1$ for $th_\text{low}$ and $th_\text{high}$ result in a sensitive change detection procedure.
The reverse results in less sensitive methods.
The correct values for the intended sensitivity need to be empirically determined.

\begin{figure}
  \centering
    \includegraphics[width=\textwidth,height=\textheight,keepaspectratio]{./Figures/chapter4/signal_threshold.pdf}
  \caption[Thresholding]{Abstract illustration of the radius ratio thresholding. The peak at $t_1$ generates a single change point. The region between $t_2$ and $t_3$ generates multiple change points, since the value of $\hbar_t$ keeps increasing.}
  \label{fig:tresholding}
\end{figure}

\subsection{Post-processing}
In our method we use the ratio based thresholding, also used by Camci~\cite{camci2010change}.
From preliminary experiments we observed that a single change in the data can cause many detected change points, using the ratio based method.
This phenomenon can be explained by the nature of the thresholding method.
Since a single (high or low) threshold is set for ratio of the radii, high values are considered to be a change point before the highest point of the peak.
As illustrated in \Cref{fig:tresholding}, the peak at $t_1$ represents a single change point.
The region from $t_2$ to $t_3$ is strictly increasing.
At $t_2$ a change point is detected, since $\hbar_t$ exceeds the threshold.
But since it is increasing up to $t_3$, all values of $\hbar_t$ in that period will trigger the ratio based method to indicate a change point.

To overcome this problem, we apply a post-processing method on the generated change points.
All the change points that are within a time period $\delta$ of each other are merged together.

In the following two chapters we will take a detailed look at the used data sets and apply the method as described in this method to those sets.