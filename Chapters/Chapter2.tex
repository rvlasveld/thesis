% !TEX root = ../main.tex
% Chapter 2

\chapter{Literature review}

\label{Chapter2} % For referencing the chapter elsewhere, use~\ref{Chapter2}

\lhead{Chapter 2. \emph{Literature review}} % This is for the header on each page - perhaps a shortened title

%----------------------------------------------------------------------------------------
\section{Outline}
\emph{Not intented for the reader.}
\begin{itemize}
  \item Literature review about Temporal Segmentation (previous draft was more about classification)
  \item Consider methods for the context of filter-methods for classification
  \item Take a loot at 3-4 different kind of methods for change detection:
    \begin{itemize}
      \item Introduction with a lot of techniques
      \item Explain why look at a few
      \item CUSUM - or other more traditional methods
      \item Density-ratio estimation
      \item Support Vector Machines (?) - if there are more sources about this
      \item (Dimensionality reduction) --> probably not
      \item \emph{Not to much attention to all techniques, focus is on SVM}
    \end{itemize}
  \item With each method, shortly look at characteristics, strengths and weaknesses and consider applicability to accelerometer sensor data
\end{itemize}


\section{Statistical framework}\label{statistical-framework}
Many applications require the detection of time points at which the underlying properties of a system change.
This problem has received a lot of attention in the fields of data mining, etc... \TODO{list and refs}.
Often this problem is formulated in a statistical framework, by inspecting the underlying data generating \gls{pdf} of the time series data.
A change point is then defined as a significant change in the properties of the \gls{pdf}, such as the mean and variance.

The widely used \gls{cusum} method by Basseville \etal \cite{basseville1993detection} follows this approach.
It originates from control methods for detection from benchmarks.
This method and some derivatiges are discussed and analyzed in Section~\ref{cusum}.

Many methods rely on pre-specified parametric model assumptions and considers the data be independent over time, which makes it less flexible to real-world applications.
The methods proposed by Kawahara \etal \cite{kawahara2009change} and Lui \etal \cite{liu2013change} try to overcome these problems by estimating the \emph{ratio} between the \glspl{pdf}, instead of estimating each \gls{pdf}.
This approach is discussed and analyzed in Section~\ref{density-ratio}.

The density-estimation methods rely on the log-likelihood ratio between \glspl{pdf}.
The method of Camci \cite{camci2010change} follows an other approach within the statistical framework, by using a \gls{svm}.
One problem it tries to overcome is the (claimed) weakness of many methods to detect a decrease in variance.
The method represents the distribution over the data points as a hyper-sphere in a higher dimension using the kernel trick.
A change in the \gls{pdf} is represented by a change in the radius of this sphere.
Section~\ref{sec:svm_in_change_detection} discusses the \gls{svm}-methods.
\TODO{Put emphasis that this is the source of inspiration for the chosen method?}

% The final method under consideration, change detection via (intrinsic) dimensionality reduction, follows a different point of view.
% Opposed to the other discussed methods, dimensionality reduction is framed in the MDL (Minimum Description Length) framework.
% It uses the estimated underlying number of parameters of the time series as a model for change detection.
% Section~\ref{dim-reduction} discuses this method.

\TODO{Maybe add dimensionality reduction, but for now leave out}



%----------------------------------------------------------------------------------------
% CUSUM and GLR
% !TEX root = ../../main.tex
\section{CUSUM}\label{cusum}

% *** read book Basseville \cite{basseville1993detection} ***


% Papers:
% \begin{itemize}
%   \item Use of Cumulative Sums of Squares for Retrospective Detection of Changes of Variance, 1994, 162 refs \cite{inclan1994use} \\
%   Implemented this algorithm.
%   \item An adaptive CUSUM-based test for signal change detection, 2006, 15 refs. \cite{alippi2006adaptive}
%   \item The MOSUM of squares test for monitoring variance changes \cite{hsu2007mosum}
% \end{itemize}

Non-Bayesian change detection algorithm (thus: no prior distribution believe available for the change time).
The \gls{cusum} method is developed by Page \cite{page1954continuous} for the application of statistical quality control (it is also known as a control chart).
Primary for detection of mean shift.
The \gls{mosum} of squares test for monitoring variance changes \cite{hsu2007mosum}.
Use of Cumulative Sums of Squares for Retrospective Detection of Changes of Variance \cite{inclan1994use}

An often used approach in the statistical framework of change detection is the \gls{cusum} as introduced by Page \cite{page1954continuous}.
Originally used for quality control in production environments, its main function is to detect change in the mean of measurements and has been applied to this problem \cite{basseville1993detection}.
It is an non-Bayesian method and thus makes no assumptions (or: prior belief distributions) for the change points.
Many extensions to this method have been proposed.
Some focus on the change in mean, such the method of Alippi and Roveri \cite{alippi2006adaptive}.
Others apply the method the problems in which the change of variance is under consideration.
Among others are there the centered version of the cumulative sums, introduced by Brown, Durbin and Evans \cite{brown1975techniques} and the \gls{mosum} of squares by \cite{hsu2007mosum}.

The method of Incl\'{a}n and Tiao \cite{inclan1994use} builds on the centered version of \gls{cusum} \cite{brown1975techniques} to detect changes in variance.
Using the \gls{icss} algorithm they are able to find multiple change points in a reflective manner.
Let $C_k = \sum_{i=1}^k \alpha_t^2$ be the cumulative sum of squares for a series of uncorrelated random variables $\{\alpha_t\}$ of length $T$.
The centered (and normalized) sum is squares is defined as
\begin{equation}
  \begin{aligned}
  D_k = \frac{C_k}{C_T} - \frac{k}{T}, & & k = 1, \dots, T, & & \text{with } D_0 = D_T = 0.
  \end{aligned}
\end{equation}
For a series with homogeneous variance, the value of $D_k$ will oscillate around $0$.
When there is a sudden change, the value will increase and exceed some predefined boundary with high probability.
The behavior of $D_k$ is related a Brownian bridge.
By using an iterative algorithm, the method is able to minimize the masking effect of successive change points.

One of the motivations for the \gls{icss} algorithm was the heavy computational burden involved with Bayesian methods, which need to calculate the posterior odds for the log-likelihood ratio testing.
The \gls{icss} algorithm avoids applying a function at all possible locations, due to the iterative search.
The authors recommend the algorithm for analysis of long sequences.

%----------------------------------------------------------------------------------------
% Density Ratio Estimation
% !TEX root = ../../main.tex
\section{Change-detection by Density-Ratio Estimation}

Formulate the problem of detecting change in the statistical framework.
Consider the probability distributions from which two consecutive segments of time series around a target time point are generated.
When the disitrubtions differ significantly the target time point is regarded as a change point.




CUSUM (cumulative sum) \cite{basseville1993detection} and GLR (generalized likelihood ratio)


The distribution over the values of time series data can be represented with a probability density function (pdf).
Two sections of a time series data can be generated with the same underlying pdf or each with a different.


%----------------------------------------------------------------------------------------
% Temporal Segmentation
% !TEX root = ../../main.tex
\section{Temporal Segmentation}\label{sec:temporal_segmentation}

\begin{itemize}
  \item Given overview of segmentation techniques, for times series data
  \item Use different ``point-of-views'', or terms
  \item ``Segmentation''
  \item ``Change detection''
  \item ``Novelty detection''
  \item Specific view on \gls{svm}s
\end{itemize}


--- Segmentation ---

``Time series segmentation for context recognition in mobile devices'' \cite{himberg2001time}. 158, 2001 \\

``An Adaptive Approach for Online Segmentation of Multi-Dimensional Mobile Data'' \cite{guo2012adaptive}. 4, 2012 \\

``Joint segmentation of multivariate time series with hidden process regression for human activity recognition'' \cite{chamroukhi2013joint}. 2013. See \ref{sec:appendix-C-joint-segmentation}. \\

``Segmentation and Recognition of Motion Streams by Similarity Search'' \cite{li2007segmentation}. 29, 2007 \\

``Novel Online Methods for Time Series Segmentation'' \cite{liu2008novel}. 22, 2008 \\

``An online algorithm for segmenting time series'' \cite{keogh2001online}. See \ref{sec:appendix-C-online-keogh}. 538, 2001 \\
``Segmenting time series: A survey and novel approach'' \cite{keogh2004segmenting}. 242, 2004 \\

``Distributed Segmentation and Classification of Human Actions Using a Wearable Motion Sensor Network'' \cite{yang2008distributed}. 44, 2008 \\

``An Automatic Segmentation Technique in Body Sensor Networks based on Signal Energy'' \cite{guenterberg2009automatic}. 14, 2009 \\

``Online Segmentation of Time Series Based on Polynomial Least-Squares Approximations'' \cite{fuchs2010online}. 24, 2010 \\

``Aligned Cluster Analysis for Temporal Segmentation of Human Motion'' \cite{zhou2008aligned}. 63, 2008 \\

--- Change detection ---

``A unifying framework for detecting outliers and change points from time series'' \cite{takeuchi2006unifying}. 87, 2006 \\

``Bayesian online changepoint detection'' \cite{adams2007bayesian}. 85, 2007 \\

``An adaptive cusum-based test for signal change detection'' \cite{alippi2006adaptive}. 18, 2006 \\

``An efficient algorithm for estimating a change-point'' \cite{cheng2009efficient}. 5, 2009 \\

``The marginalized likelihood ratio test for detecting abrupt changes'' \cite{gustafsson1996marginalized}. 80, 1996 \\

``The MOSUM of squares test for monitoring variance changes'' \cite{hsu2007mosum}. 4, 2007 \\

``Use of cumulative sums of squares for retrospective detection of changes of variance'' \cite{inclan1994use}. 643, 1994 \\

``Change-point detection in time-series data by direct density-ratio estimation'' \cite{kawahara2009change}. 52, 2009 \\
``Sequential change-point detection based on direct density-ratio estimation'' \cite{kawahara2012sequential}. 22, 2012 \\
``Change-point detection in time-series data by relative density-ratio estimation'' \cite{liu2013change}. 11, 2013 \\

``Change point detection in time series data using support vectors'' \cite{camci2010change}. 3, 2010 \\

--- Novelty detection ---

``Online novelty detection on temporal sequences'' \cite{ma2003online}. 146, 2003 \\
``Time-series novelty detection using one-class support vector machines'' \cite{ma2003time}. 78, 2003 \\

``Novelty detection: a review—part 1: statistical approaches'' \cite{markou2003novelty}. 697, 2003 \\

``Support Vector Method for Novelty Detection'' \cite{scholkopf1999support}. 337, 1999 \\

--- SVMs ---

``Change point detection in time series data using support vectors'' \cite{camci2010change}. 3, 2010 \\

``Time-series novelty detection using one-class support vector machines'' \cite{ma2003time}. 78, 2003 \\

``Support Vector Method for Novelty Detection'' \cite{scholkopf1999support}. 337, 1999 \\

``Support vector domain description'' \cite{tax1999support}. 907, 1999
``Support vector data description applied to machine vibration analysis'' \cite{tax1999supportdata}. 76, 1999 \\

``Human Activity Recognition on Smartphones using a Multiclass Hardware-Friendly Support Vector Machine'' \cite{anguitahuman}. 13, 2012 \\

``A geometric approach to support vector machine (SVM) classification'' \cite{mavroforakis2006geometric}. 136, 2006 \\

``Support vector machines in remote sensing: A review'' \cite{mountrakis2011support}. 150, 2011 \\

``Predicting time series with support vector machines'' \cite{muller1997predicting}. 693, 1997 \\



%----------------------------------------------------------------------------------------
% Support Vector Machines
% !TEX root = ../../main.tex
\section{Support Vector Machines in Change Detection}\label{sec:literature_review_svm}

Many proposals in the field of remote sensing and \gls{har} make use of \glspl{svm} as a (supervised) model learning method.
An elaborate overview of applications is given in \cite{mountrakis2011support}.
In this section we review methods of change detection based on the applications of \glspl{svm} as model construction method.
A number of proposals have been made using this method.
Sch{\"o}lkopf \etal \cite{scholkopf1999support} applies Vapnik's principle, to never solve a problem which is more general than the one that one is actually interested in, to novelty detection.
In the case of novelty detection, they argue there is no need for density estimation of the distribution.
Simple algorithms estimate the density by considering how many data points fall in a region of interest.
The \gls{nu-svm} method instead starts with the number of data points that should be within the region and estimates a region with that desired property.
It builds on the method of Vapnik and Vladimir \cite{vapnik1963pattern}, which characterizes a set of data by separating it from the origin.
The \gls{nu-svm} method adds the kernel method, allowing non-linear decision functions, and incorporates `softness' by the $\nu$-parameter.
Whereas the method in \cite{vapnik1963pattern} focuses on two-class problems, the \gls{nu-svm} method introduces the method the one-class problems.

The method introduced by Ma and Perkins \cite{ma2003time} creates a \emph{projected phase} of a time series data, which is intuitively equal to applying a high-pass filter to the time series.
The projected phase of the time series combines a history or data points to a vector, which are then classified by the \gls{nu-svm} method.
The method is applied to a simple synthetic sinusoidal signal with small additional noise and a small segment with large additional noise.
The algorithm is able to detect that segment, without false alarms.

The algorithms of \glspl{svm} has been applied to the problem of \gls{har}, as by Anguita \etal \cite{anguita2012human}.
In that research a multi-class classification problem is solved using \glspl{svm} and the One-Vs-All method.
The method exploits fixed-point arithmetic to create a hardware-friendly algorithm, which can be executed on a smartphone.\footnote{Smartphones have limited resources and thus require energy-efficient algorithms. In \cite{anguita2007hardware} a Hardware-Friendly \gls{svm} is introduced. It uses less memory, processor time and power consumption, with a loss of precision.}

With the same concepts as \gls{nu-svm}, the \gls{svdd} method by Tax and Duin \cite{tax1999support,tax2004support} uses a separating hypersphere (in contrast with a hyperplane) to characterize the data.
The data points that lie outside of the created hypersphere are considered to be outliers, which number is a pre-determined fraction of the total number of data points.

The method by Yin \etal \cite{yin2008sensor} uses the \gls{svdd} method in the first phase of a two-phase algorithm to filter commonly available normal activities.
The \gls{svcpd} method of Camci \cite{camci2010change} uses \gls{svdd} applies it to time-series data.
By using an \gls{oc-svm} the method is able to detect changes in mean variance.
Especially the detection of variance \emph{decrease} is an improvement over other methods that rely in the \gls{kl} divergence, since these are unable to detect decreases in variance \cite{takeuchi2006unifying}.
This main research subject of this thesis is to apply the method of Camci \cite{camci2010change} to sensor data (such as accelerometer time series) obtained by on-body smartphones.

The inner workings of \glspl{oc-svm} are discussed in detail in \Cref{sec:one_class_svm}.

\TODO{Add notion of `supervised' vs `unsupervised' learning. \gls{oc-svm} is unsupervised.}


% \section{Change-detection by Support Vector Machines}\label{svm}
% \TODO{REMOVE THIS SECTION?}

% Introduced by Cortes and Vapnik \cite{vapnik1998statistical, vapnik1999nature}, Support Vector Machines offer a way to segment, and classify, linear separable data.
% This technique can also be applied to estimate density functions of given time series \cite{weston1999support}.
% When combined with a mapping function, which maps the data from the input space $I$ to a higher dimension feature space $F$, the input data can be non-linear separable.
% A separating linear hyperplane that segments the data in the feature space $F$, yields a non-linear segmentation in the lower-dimensional input space $I$.
% Instead of explicitly mapping the input data to the higher dimensional space, a kernel function can be used.
% This kernel function can calculate values of the feature space directly, without the need to first map the input values to this space.
% This process is referred to as the kernel trick.

% The proposed method of Camci \cite{camci2010change} uses a one-class support vector machine to segment time series data.
% One-class SVMs are used to describe the current data under consideration, by assuming all data points are from the same class \cite{tax2001one}.
% To cope with possible errors or outliers a soft-margin is applied \cite{cortes1995support}.
% The class is described by a spherical boundary around the data with center $c$ and radius $r$, such that the volume is minimized.
% New data points are consecutively mapped from the input space to the feature space.
% The retrieved (high dimension) data point can be in- or outside the earlier constructed hyper-sphere, thereby giving information about a possible change point.
% \TODO{NOTE: using the in/out position of a new datapoint is different from the approach in this thesis, by only using the model properties.}
% Following the definition of Camci \cite{camci2010change}, the class description is obtained by minimizing $r^2$:
% \begin{equation}
%   \mathrm{Min}\ r^2
% \end{equation}
% \begin{equation}
%   \mathrm{Subject\ to} : \|\mathbf{x}_i - \mathbf{c}\|^2 \le r^2\ \forall i,\ \mathbf{x}_i : i \mathrm{th\ data\ point}
% \end{equation}

% To be able to handle outliers in the input data, a penalty cost function $\varepsilon_i$ for each outlier can be added.

% Using this one-class SVM formulation, differences between two (consecutive) windows of data points with size $w$ can be obtained.
% The first window is used as the input set, $h_1$ and the second as the test set $h_t$.
% For the first window a one-class SVM is constructed, yielding in a representation by $c_1$ and $r_1$.
% When the data points of the second window belong to the same class, the representation of that one-class SVM would equal the first:
% \begin{equation}
% \begin{aligned}
%   c_1 = c_2, & &  r_1 = r_2
% \end{aligned}
% \end{equation}

% In case the second window of data points does not belong to the same class, \ie the probability density function that describes the data differs from the first, the describing values of the second window will differ from the first.
% A difference in the SVM center $c$ or radius $r$ represent a change in the mean and variance, respectively.
% The amount of difference can be expressed by a dissimilarity measure over the representations.
% When the dissimilarity between the two windows exceeds some predefined threshold $th$, there exists a change point between the windows.

% \TODO{Give visual explanation with circles and in- and out-side new data points}

% \TODO{Make clear what approach we use (model properties) and create link/bridge to next Chapter (3), which discusses SVM in detail.}

%----------------------------------------------------------------------------------------
% Dimensionality reduciton
% % !TEX root = ../../main.tex
\section{Change-detection by Dimensionality Reduction / Covariance structure}\label{dim-reduction}


