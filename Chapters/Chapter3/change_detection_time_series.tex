% !TEX root = ../../main.tex
\section{Problem Formulation of Change Detection}\label{sec:change_detection_time_series}
% \TODO{Write section introduction} \\
% \emph{The section should relate the problem of outlier detection to change detection.
% It should transform the traditional problem formulation to that of time series data.
% Build to one-class classification, which can be used for outlier detection}

% \subsection{Relation to outlier detection}
The problems of outlier and novelty detection, segmentation, and change detection in time series data are closely related.
The terminology depends on the field of application, but there are subtle differences.
The problem of outlier detection is concerned with finding data objects in a data set which have small resemblance with the majority of the data objects.
These objects can be regarded as erroneous measurements.
In the case of novelty detection these objects are considered to be member of a new class of objects.
The unifying framework of Takeuchi and Yamanishi \cite{takeuchi2006unifying}, \gls{changeFinder}, creates a two stage process expressing change detection in terms of outlier detection.
The first stage determines the outliers in a time series by giving a score based on the deviation from a learned model, and thereby creates a new time series.
The second stage runs on that new created time series and calculates a average over a window of the outlier scores.
The problem of change detection is then reduced to outlier detection over that average-scored time series.
The implementation by Camci \cite{camci2010change}, \gls{svcpd}, implements outlier detection with the \gls{svdd} algorithm to detect changes in mean and variance.

% \subsection{Problem formulation}\label{subsec:change_detection_problem_formulation}
The problem of change point detection can be formulated using different type of models, as discussed in~\ref{subsec:segmentation}.
The methods by Takeuchi and Yamanishi \cite{takeuchi2006unifying} and Camci \cite{camci2010change} use the following formulation for change detection, which we will also use for our problem formulation.
The algorithm searches for sudden changes in the time series data.
In other words, slowly changing properties in the data are not considered to be changes.
Considered a time series $x_1, x_2, \dots$, which is drawn from a stochastic process $p$.
Each $x_t$ (t = 1, 2, \dots) is a $d$-dimensional real valued vector.
Assume $p$ can be ``decomposed'' in two different \gls{iid} stationary one-dimensional Gaussian processes $p^{(1)}$ and $p^{(2)}$.
For a time point $a$ data points for which $t < a$ are drawn from $p^{(1)} = N(\mu_1, \sigma_1^2)$ and for $t \geq a$ from $p^{(2)} = N(\mu_2, \sigma_2^2)$.
If $p^{(1)}$ and $p^{(2)}$ are different, then the time point $t = a$ is a \emph{change point}.
Takeuchi and Yamanishi~\cite{takeuchi2006unifying} expess the similarity between the stochastic processes with the \gls{kl} divergence $D(p^2||p^1)$.
It is observed that their method is not able to detect a change by decrease in variance \cite{takeuchi2006unifying,camci2010change}.
This problem is the motivation for Camci \cite{camci2010change} to create the \gls{svcpd} algorithm.

The definition of change point being sudden changes in the time series data is in line with the search of changes in activities.
Since we are only interested in different activities (which are represented by sudden changes), slight changes within an activity are not of interest.

% -- Literature --

% ``Online segmentation of time series based on polynomial least-squares approximations'', \cite{fuchs2010online}. 25, 2010 \\

% \subsection{General framework for outlier detection}
% In \cite{schubert2012local} a general framework for outlier detection is proposed.
% The authors compare existing methods and identified the common building block of the algorithms.
% The focus is on unsupervised methods, using information from a local selection of data objects for the detection of outliers.
% The following common algorithmic building blocks are identified:

% \begin{enumerate}
%   \item Context: a ``local'' context of an object $o$ for model building
%   \item Model: the method used for building the model
%   \item Reference: a ``reference'' context of object $o$ for model comparison
%   \item Comparison: the method used for model comparison
%   \item Normalization: a (global) normalization procedure
% \end{enumerate}

% Here the \emph{context} and \emph{reference} are sets of objects used for model building and model comparison, respectively.


% \subsection{Regression, Classification, etc}

% General framework of outlier detection, change detection in context of (simple?) regression, classification.

% \subsection{M-estimators}
% \TODO{Move this section to somewhere in the end; is specification/extension, not base material}
% To make a method less sensitive to outliers (in the training data) techniques from robust statistics are applied.
% The term \emph{robustness} has many interpretations, one of them that it ``signifies insensi­tivity to small deviations from the [prior] assumptions [about the underlying situation]'', according to Huber \cite{huber2009robust}.
% Methods have been proposed to apply robust statistics in the form of M-estimators to \glspl{svm}, such as \cite{choi2009least,chen2004m,suykens1999least}.

% \TODO{Question: do slack-variables (allowance of outliers) make SVM robust?}