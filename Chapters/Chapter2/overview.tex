% !TEX root = ../../main.tex
\section{Overview}\label{sec:literature_review_overview}
As described in \Cref{Chapter1}, the focus of this research is the application of temporal segmentation to accelerometer data.
This is with the assumption that explicit temporal segmentation can aid the process of temporal classification.

\TODO{Give examples of temporal classification techniques}

In recent years a lot is research is performed in the classification of human activities, recorded by on-body sensors.
Especially since smartphones with accelerometer sensors became more available, research has focused on recordings from these devices.

\TODO{Include highlights from old-classification text}

In the following of this chapter we will start with an overview of statistical models used in this field of study.
That section will act as a starting point to look at the different types, applications and methods of temporal segmentation.
The following section, \Cref{sec:literature_review_temporal_segmentation} follows the distinction made and looks at a collection of temporal segmentation and change detection methods.
It will also relate the concept of \gls{occ} to temporal segmentation and change detection.
The final section of this chapter, \Cref{sec:literature_review_svm}, will discuss a specific class of algorithms for \gls{occ}, namely the methods that use \glspl{svm} for the classification task.

% In the following sections of this is chapter we begin with a review of current literature on temporal segmentation of time series data.
% Many techniques of temporal segmentation rely on change detection algorithms, implemented as outlier or novelty detection.
% In \Cref{sec:change_detection_methods} we will give an overview of algorithms that detect changes in time series data, based on model assumptions over the data.
% We end this chapter with a review on \gls{svm} based methods and algorithms, which will serve as a basis of this research.

% \TODO{Note to self: methods based on model-fitting do not directly use change points}

% \TODO{Plek bedenken waar SVM en Dimensionality Reduction aan elkaar gekoppeld kunnen worden. Ergens in chapter 3?}