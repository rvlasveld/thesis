% !TEX root = ../../main.tex
\section{Change-detection by Density-Ratio Estimation}\label{sec:density-ratio}

% Papers:
% \begin{itemize}
%   \item Change-Point Detection in Time-Series Data by Direct Density-Ratio Estimation, 2009, 45 refs \cite{kawahara2009change}
%   \item Change-Point Detection in Time-Series Data by Relative Density-Ratio Estimation, 2013, 3 refs \cite{liu2013change}
%   \item Density ratio estimation in machine learning (Book), 2009, 24 refs \cite{sugiyama2012density}
%   \item Direct importance estimation for covariate shift adaptation, 2008, 84 refs \cite{sugiyama2008direct}
% \end{itemize}

% Formulate the problem of detecting change in the statistical framework.
% Consider the probability distributions from which two consecutive segments of time series around a target time point are generated.
% When the disitrubtions differ significantly the target time point is regarded as a change point.




% CUSUM (cumulative sum) \cite{basseville1993detection} and GLR (generalized likelihood ratio)


% The distribution over the values of time series data can be represented with a probability density function (pdf).
% Two sections of a time series data can be generated with the same underlying pdf or each with a different.

Many approaches to detect change points monitor the logarithm of the likelihood ratio between two consecutive intervals.
A change point is regarded to be the moment in time when the underlying probabilistic generation function changes.
Some methods which rely on this are novelty detection, maximum-likelihood estimation and online learning of autoregressive models \cite{kawahara2009change}.
A limitation of these methods is that they rely on pre-specified parametric models.
Non-parametric models, for which the number and nature of the parameters are undetermined, for density estimation have been proposed, but it is said to be a hard problem \cite{hardle2004nonparametric, sugiyama2012density}.
A solution to this is to estimate the \emph{ratio} of probabilities instead of the probabilities themselves.
One of the recent methods to achieve this is the \gls{kliep} by Sugiyama \etal \cite{sugiyama2008direct}.

The proposed method by Kawahara and Sugiyama~\cite{kawahara2009change} uses an online version of the \gls{kliep} algorithm.
It considers \emph{sequences} of samples (rather than samples directly) because the time series samples are generally not independent over time.
The method detects change by monitoring the logarithm of the likelihood ratio between densities of reference (past) and test (current) time intervals.
If it exceeds a predetermined threshold value, the beginning of the test interval is marked as a change point.

Since the density ratio is unknown, it needs to be estimated.
The naive approach is to estimate it using estimated densities of the two intervals.
Since this is known to be a hard problem and sensitive for errors, the solution would be to estimate the ratio directly.

The method by Liu \etal \cite{liu2013change} estimates the ratio of probabilities directly, instead of estimating the densities explicitly.
Other methods using ratio-estimation are the Unconstrained Least-Squares Importance Fitting (uLSIF) method \cite{kanamori2009least}, and an extension which possesses a superior non-parametric convergence property: Relative uLSIF (RuLSIF) \cite{yamada2013relative}.