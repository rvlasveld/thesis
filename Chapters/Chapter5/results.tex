% !TEX root = ../../main.tex
\section{Results}\label{sec:artificial_data_results}
For an objective measure of quality of the proposed method applied to the four aforementioned data sets, we employ two metrics, as defined by Takeuchi \etal \cite{takeuchi2006unifying}.
The first metric is the $\operatorname*{average \_ benefit}$, for which we first need to define the notion of $\operatorname*{benefit}$ for each change:
\begin{equation}\label{eq:benefit}
  \operatorname*{benefit}(y) = 1 - |t_y - t_y^*| / 10  \mbox{~~~(when } |t_y - t_y^*| < 10 \mbox{)},
\end{equation}
where $t_y$ is the time of the discovered change point and $t_y^*$ is the true change time.
We can then define the $\operatorname*{average \_ benefit}$ over all the discovered change points $Y$ as:
\begin{equation}\label{eq:average_benefit}
  \operatorname*{average \_ benefit}(Y) = \frac{\sum_{y=1}^Y \operatorname*{benefit}(y)}{Y}.
\end{equation}
Besides the benefit, which rewards discovered change points close to the actual change points, we need to penalize discovered change points which do not relate to a true change point.
For this we use the \acrlong{far}, expressed as
\begin{equation}\label{eq:false_alarm_rate}
  \operatorname*{far}(Y) = \frac{Y_{\operatorname*{false}}}{Y}.
\end{equation}

For a high quality method the $\operatorname*{average\_benefit}$ should be high and the $\operatorname*{far}$ should be low.

\TODO{Create plots of FAR and average benefit, as a kind of ROC curve.}

\begin{table}
  \centering
  \begin{tabulary}{\textwidth}{|l|c|c|c|c|}
    \cline{2-5}
    \multicolumn{1}{l|}{} & Set 1 & Set 2 & Set 3 & Set 4 \\
    \hline
    High & 1.6 & 1.6 & 1.5 & 2.2 \\
    \hline
    Low & 0.1 & 0.1 & 0.5 & 0.1 \\
    \hline
    Closeness & 10 & 10 & 10 & 50 \\
    \hline
    Number of change points detected & 10 & 11 & 11 & 11 \\
    \hline
    Sum of offsets & 25 & 68 & 169 & 7 \\
    \hline
    Window & 50 & 100 & 50 & 50 \\
    \hline
    Sigma & 13 & 13 & 15 & 13 \\
    \hline
  \end{tabulary}
  \caption[Results artificial data sets]{Results of the artificial data sets.}
  \label{tab:results_artificial}
\end{table}

\TODO{Instead of offset, use mean and spread measure, in \Cref{tab:results_artificial}.}